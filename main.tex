\documentclass[12pt,a4paper]{report}
\setlength{\headheight}{15.35403pt}
\addtolength{\topmargin}{-3.35403pt}

\usepackage{amsmath}
\usepackage[utf8]{inputenc}
\usepackage[T1]{fontenc}
\usepackage[brazil]{babel}
\usepackage{lmodern}
\usepackage{geometry}
\geometry{margin=2.5cm}

\setlength{\parindent}{1em}

\usepackage{booktabs}
\usepackage{longtable}
\usepackage{hyperref}
\usepackage{graphicx}
\usepackage{indentfirst}
\usepackage{background}
\usepackage{fancyhdr}
\usepackage{lastpage}
\usepackage[acronym,toc]{glossaries}
\usepackage{array}
\usepackage{float}
\usepackage{microtype}

% Definindo as siglas:
\newacronym{tap}{TAP}{Termo de Abertura do Projeto}
\newacronym{gpf}{GPF}{Gestão de Projetos em Fundações}
\newacronym{fundep}{Fundep}{Fundação de Desenvolvimento da Pesquisa}
\newacronym{fiotec}{Fiotec}{Fundação para o Desenvolvimento Científico e Tecnológico em Saúde}
\newacronym{ufmg}{UFMG}{Universidade Federal de Minas Gerais}
\newacronym{fiocruz}{Fiocruz}{Fundação Oswaldo Cruz}
\newacronym{rabbitmq}{RabbitMQ}{software de mensageria}
\newacronym{cia}{CIA}{Centro Integrado de Atendimento}
\newacronym{bhs}{BHS}{empresa terceirizada para contratação de pessoal}

\backgroundsetup{
  scale=1,
  color=black,
  opacity=1,
  angle=0,
  contents={\includegraphics[width=\paperwidth,height=\paperheight]{img/Timbrado.jpg}}
}

\title{Proposta Comercial para o Período de 90 (noventa) dias}
\author{
  Elaborado por: André Vinícius de Oliveira\\
  Aprovado por: Carlos Henrique de Morais Bomfim\\
  Aprovado por: Walmir Matos Caminhas
}
\date{21/05/2025}

\pagestyle{fancy}
\fancyhf{}
\fancyhead[R]{Página \thepage\ de \pageref{LastPage}}
\fancyhead[L]{\leftmark}

\makeatletter
\let\ps@plain\ps@fancy
\makeatother

\begin{document}

\chapter*{Anexo I – Proposta Comercial para o Período de 90 (noventa) dias}
\addcontentsline{toc}{chapter}{Anexo I – Proposta Comercial para o Período de 90 (noventa) dias}

\section*{1. Objetivo}

Este anexo apresenta a estimativa de custos para a fase inicial do projeto, correspondente a 90 (noventa) dias de execução do pré-projeto do \gls{gpf}. Para efeito de estimativa, consideramos início em 02 de junho de 2025. Esses valores podem variar proporcionalmente caso a data de início seja diferente, uma vez que a cobrança de terceiros é feita por hora trabalhada.

\section*{2. Disposições Gerais}

Os cargos previstos nesta fase poderão ser ocupados por profissionais das fundações \gls{fundep} e \gls{fiotec} ou, conforme necessidade e disponibilidade interna, por meio da contratação de terceiros. Os valores aqui apresentados servem como base de planejamento e poderão variar de acordo com as condições específicas de contratação, observando os intervalos informados.

As cargas horárias foram calculadas com base nos \textbf{64 dias úteis} correspondentes ao período de \textbf{02 de junho de 2025 a 29 de agosto de 2025}, segundo o calendário de \textbf{Belo Horizonte (MG)}, considerando a semana de \textbf{segunda à sexta}, descontando os feriados de 19/06 (Corpus Christi – ponto facultativo) e 15/08 (Assunção de Nossa Senhora).

\section*{3. Estimativa de Custos por Cargo}

\begin{table}[H]
    \centering
    \small
    \resizebox{\textwidth}{!}{
        \begin{tabular}{@{}llrrrr@{}}
            \toprule
            \textbf{Cargo} & \textbf{Qtde} & \textbf{Horas} & \textbf{Valor Hora (R\$)} & \textbf{Custo Mínimo (R\$)} & \textbf{Custo Máximo (R\$)} \\
            \midrule
            Coordenador de Projeto & 1 & 512 & 89 -- 182 & 45.568,00 & 93.184,00 \\
            Subcoordenadores de Projeto & 2 & 512 & 89 -- 182 & 91.136,00 & 186.368,00 \\
            Consultor Técnico Sênior & 1 & 48 & 400 & 19.200,00 & 19.200,00 \\
            Product Owner / Analista de Negócio & 7 & 512 & 112 -- 168 & 400.384,00 & 600.576,00 \\
            UX/UI Designer & 1 & 256 & 57 -- 135 & 14.592,00 & 34.560,00 \\
            Arquiteto de Software & 2 & 256 & 135 -- 200 & 69.120,00 & 102.400,00 \\
            Documentador & 1 & 512 & 75 -- 178 & 38.400,00 & 91.136,00 \\
            Representantes do Negócio & 30 & 256 & — & — & — \\
            \bottomrule
        \end{tabular}
    }
    \caption{Estimativa de Custo por Cargo}
    \label{tab:estimado-por-cargo}
\end{table}

\section*{4. Totais Estimados}

A Tabela \ref{tab:totais-estimados} apresenta o custo total estimado para os 90 dias de pré-projeto, considerando que todos os profissionais listados serão contratados como terceiros (100\% contratação terceirizada). Esses valores contemplam apenas o custo direto com as horas de trabalho e não incluem encargos adicionais (administrativos, tributos, gestão de contrato, etc.), que deverão ser negociados separadamente entre as fundações.

\begin{table}[H]
    \centering
    \small
    \resizebox{\textwidth}{!}{
        \begin{tabular}{@{}p{6.5cm}cc@{}}
            \toprule
            \textbf{Tipo de Custo} & \textbf{Valor Total Mínimo (R\$)} & \textbf{Valor Total Máximo (R\$)} \\
            \midrule
            Custo Estimado & \textbf{678.400,00} & \textbf{1.127.424,00} \\
            \bottomrule
        \end{tabular}
    }
    \caption{Totais estimados de custo considerando 100\% de contratação terceirizada.}
    \label{tab:totais-estimados}
\end{table}

\section*{5. Observações}

\begin{itemize}
    \item A composição final das equipes será definida em comum acordo entre as fundações, podendo sofrer ajustes ao longo da execução.
    \item Os representantes do negócio são colaboradores internos das fundações e não gerarão custos diretos nesta etapa.
\end{itemize}


\end{document}
