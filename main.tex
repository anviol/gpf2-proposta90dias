\documentclass[12pt,a4paper]{report}
\setlength{\headheight}{15.35403pt}
\addtolength{\topmargin}{-3.35403pt}

\usepackage{amsmath}
\usepackage[utf8]{inputenc}
\usepackage[T1]{fontenc}
\usepackage[brazil]{babel}
\usepackage{lmodern}
\usepackage{geometry}
\geometry{margin=2.5cm}

\setlength{\parindent}{1em}

\usepackage{booktabs}
\usepackage{longtable}
\usepackage{hyperref}
\usepackage{graphicx}
\usepackage{indentfirst}
\usepackage{background}
\usepackage{fancyhdr}
\usepackage{lastpage}
\usepackage[acronym,toc]{glossaries}
\usepackage{array}
\usepackage{float}
\usepackage{microtype}

% Definindo as siglas:
\newacronym{tap}{TAP}{Termo de Abertura do Projeto}
\newacronym{gpf}{GPF}{Gestão de Projetos em Fundações}
\newacronym{fundep}{Fundep}{Fundação de Desenvolvimento da Pesquisa}
\newacronym{fiotec}{Fiotec}{Fundação para o Desenvolvimento Científico e Tecnológico em Saúde}
\newacronym{ufmg}{UFMG}{Universidade Federal de Minas Gerais}
\newacronym{fiocruz}{Fiocruz}{Fundação Oswaldo Cruz}
\newacronym{rabbitmq}{RabbitMQ}{software de mensageria}
\newacronym{cia}{CIA}{Centro Integrado de Atendimento}
\newacronym{bhs}{BHS}{empresa terceirizada para contratação de pessoal}

\backgroundsetup{
  scale=1,
  color=black,
  opacity=1,
  angle=0,
  contents={\includegraphics[width=\paperwidth,height=\paperheight]{img/Timbrado.jpg}}
}

\title{Proposta Comercial para o Período de 90 (noventa) dias}
\author{
  Elaborado por: André Vinícius de Oliveira\\
  Aprovado por: Carlos Henrique de Morais Bomfim\\
  Aprovado por: Walmir Matos Caminhas
}
\date{21/05/2025}

\pagestyle{fancy}
\fancyhf{}
\fancyhead[R]{Página \thepage\ de \pageref{LastPage}}
\fancyhead[L]{\leftmark}

\makeatletter
\let\ps@plain\ps@fancy
\makeatother

\begin{document}

    \chapter*{Anexo I – Proposta Comercial para o Período de 90 (noventa) dias}
    \addcontentsline{toc}{chapter}{Anexo I – Proposta Comercial para o Período de 90 (noventa) dias}

    \section*{1. Objetivo}

    Este anexo apresenta a estimativa de custos para a fase inicial do projeto, correspondente a 90 (noventa) dias de execução do pré-projeto do \gls{gpf}. Para efeito de estimativa, considerou-se uma carga de trabalho equivalente a \textbf{66 dias úteis}, estimados com base em um período de três meses típicos de trabalho, descontando feriados e finais de semana, o que permite estimar os custos com base em horas previstas de dedicação por perfil profissional. Os valores podem variar proporcionalmente caso haja alteração na quantidade de dias úteis efetivamente trabalhados, uma vez que a cobrança de terceiros é feita por hora trabalhada.

    \section*{2. Disposições Gerais}

    Os cargos previstos nesta fase poderão ser ocupados por profissionais das fundações \gls{fundep} e \gls{fiotec} ou, conforme necessidade e disponibilidade interna, por meio da contratação de terceiros. Os valores aqui apresentados servem como base de planejamento e poderão variar de acordo com as condições específicas de contratação, observando os intervalos informados.

    \section*{3. Estimativa de Carga e Valores}

    A Tabela \ref{tab:carga-e-valores} apresenta a composição da equipe prevista para a fase do pré-projeto do \gls{gpf}, com a carga horária individual, forma de alocação e valores estimados para os profissionais terceirizados. Os valores apresentados referem-se exclusivamente aos profissionais terceirizados e servem como base para planejamento orçamentário desta etapa do projeto.

    \begin{table}[H]
        \centering
        \small
        \resizebox{\textwidth}{!}{
            \begin{tabular}{@{}p{4.5cm}cclrr@{}}
                \toprule
                \textbf{Cargo} & \textbf{Qtde} & \textbf{Horas} & \textbf{Alocação} & \textbf{Custo Hora (R\$)} & \textbf{Custo Total (R\$)} \\
                \midrule
                Coordenador de Projeto              & 1  & 80  & \gls{fundep}     & --     & --         \\
                Subcoordenador de Projeto           & 1  & 528 & \gls{fiotec}     & --     & --         \\
                Subcoordenador de Projeto           & 1  & 198 & \gls{fundep}     & --     & --         \\
                Consultor Técnico Sênior            & 1  & 80  & Terceirizado & 400,00 & 32.000,00  \\
                Product Owner / Analista de Negócio & 2  & 528 & \gls{fiotec}     & --     & --         \\
                Product Owner / Analista de Negócio & 5  & 528 & Terceirizado & 168,00 & 443.520,00 \\
                UX/UI Designer                      & 1  & 528 & Terceirizado & 135,00 & 71.280,00  \\
                Arquiteto de Software               & 1  & 528 & \gls{fiotec}     & --     & --         \\
                Arquiteto de Software               & 1  & 528 & Terceirizado & 200,00 & 105.600,00 \\
                Documentador                        & 1  & 528 & Terceirizado & 178,00 & 93.984,00  \\
                Representantes do Negócio           & 15 & 198 & \gls{fundep}     & --     & --         \\
                Representantes do Negócio           & 15 & 198 & \gls{fiotec}     & --     & --         \\
                \bottomrule
            \end{tabular}
        }
        \caption{Composição da equipe, carga horária por pessoa e estimativas de custo para profissionais terceirizados.}
        \label{tab:carga-e-valores}
    \end{table}

    O valor estimado para esta fase inicial do projeto é de \textbf{R\$ 746.384,00}, correspondente ao custo direto da contratação de profissionais terceirizados conforme as funções e cargas horárias previstas.

    \textbf{Não estão incluídos nesse valor encargos indiretos}, tais como administração, tributos, gestão contratual e eventuais despesas operacionais, que deverão ser definidos em tratativas futuras entre as partes envolvidas.

    \section*{4. Observações}

    \begin{itemize}
        \item A composição final da equipe será definida em comum acordo entre \gls{fundep} e \gls{fiotec}, podendo sofrer ajustes ao longo da execução conforme necessidades operacionais.
        \item Os representantes do negócio atuarão como usuários-chave das áreas envolvidas e não gerarão custos diretos nesta fase, por serem profissionais das próprias fundações.
    \end{itemize}


\end{document}